\documentclass{article}
\usepackage[utf8]{inputenc}
\usepackage[T2A]{fontenc}
\usepackage[mytitle={Oleksandr Kulkov. ETH Zürich. Linear Algebra. Week 11},
            mylang=eng]{my_style}


\begin{document}

\paragraph{Topics of the week} 

\begin{enumerate}
    \item Determinant and its properties, definition via permutations, connection to matrix inverse, co-factors and the determinant, Cramer's rule
    \item Complex numbers, calculations with complex numbers, conversion between Cartesian form and polar form, Euler's formula
    \item Fundamental theorem of algebra, roots of polynomials
    \item Complex-valued vectors and matrices
    \item Eigenvalues and eigenvectors, definition and 2x2 examples
\end{enumerate}

\paragraph{Determinant} For a set of vectors $a_1,\dots,a_n$, define the signed volume $V(a_1,\dots,a_n)$:

\begin{enumerate}
    \item $V(\alpha v, \dots) = \alpha V(v,\dots)$;
    \item $V(a+b, \dots) = V(a,\dots) + V(b,\dots)$;
    \item $V(a,b,\dots) = -V(b,a,\dots)$;
    \item $V(e_1,\dots,e_n) = 1$.
\end{enumerate}

\textbf{Properties}:

\begin{enumerate}
    \item $\det AB = \det A \det B$ $\iff$ All spatial objects increase their volume by the factor of $\det$;
    \item $A^{-1}$ exists $\iff \det A \neq 0 \iff$ columns of $A$ belong to a smaller-dimensional subspace;
    \item $\det A = \det A^\top$;
\end{enumerate}

\paragraph{Sign of permutation} $\operatorname{sgn} \sigma$ is $(-1)^k$, where $k$ is, equivalently:

\begin{enumerate}
    \item The parity of the number of inversions in $\sigma$;
    \item The parity of the number of swaps needed to turn $\sigma$ into identity;
    \item The difference in parity between $n$ and the number of cycles in cycle presentation of $\sigma$.
\end{enumerate}

\paragraph{Leibniz formula} We can explicitly write out the determinant as

$$
\det A = \sum\limits_{\sigma \in S_n} \operatorname{sgn} \sigma \prod\limits_{i=1}^n A_{i \sigma_i}
$$

\paragraph{Cofactor matrix} of $A$ is the matrix $M$ s.t. $M_{ij}$ is $(-1)^{i+j}$ times the determinant of the matrix that is obtained from $A$ by removing its $i$-th row and $j$-th column.

\paragraph{Laplace expansion} also called cofactor expansion:

$$
\det A = \sum\limits_{j=1}^n (-1)^{i+j} A_{ij} M_{ij} = \sum\limits_{i=1}^n (-1)^{i+j} A_{ij} M_{ij},
$$

where $M$ is the cofactor matrix. Example:

$$
\begin{vmatrix}
    A_{11} & A_{12} & A_{13} \\
    A_{21} & A_{22} & A_{23} \\
    A_{31} & A_{32} & A_{33}
\end{vmatrix} = A_{11} \begin{vmatrix}
    A_{22} & A_{23} \\ A_{32} & A_{33}
\end{vmatrix} - A_{12} \begin{vmatrix}
    A_{21} & A_{23} \\ A_{31} & A_{33}
\end{vmatrix} + A_{13} \begin{vmatrix}
    A_{21} & A_{22} \\
    A_{31} & A_{32}
\end{vmatrix}
$$

\paragraph{Cramer's rule} Consider $Ax=b$. We need to find $x_1,\dots,x_n$ s.t. $b = x_1 v_1 + \dots + x_n v_n$. Then,

$$
x_i = \frac{V(v_1,\dots,v_{i-1}, b, v_{i+1}, \dots, v_n)}{V(v_1,\dots,v_{i-1}, v_i, v_{i+1}, \dots, v_n)}
$$

In other words, $x_i = \frac{\det A_i}{\det A}$, where $A_i$ replaces $v_i$ with $b$.

\paragraph{Adjugate matrix} $\operatorname{adj} A = M^\top$, where $M$ is the co-factor matrix:

\begin{enumerate}
    \item $A \operatorname{adj} A = \operatorname{adj} A A = I \det A$;
    \item If $A$ is invertible, $A^{-1} = \frac{\operatorname{adj} A}{\det A}$.
\end{enumerate}

\paragraph{Determinant in row operations} When we do Gaussian elimination,

\begin{enumerate}
    \item Multiplying a row by $\alpha$ also multiplies the determinant by $\alpha$;
    \item Adding one row, multiplied by $\alpha$, to another does \textbf{not} change the determinant;
    \item Swapping two rows multiplies the determinant by $-1$.
\end{enumerate}

\paragraph{Change of basis} What happens to a matrix when we switch from $e_1,\dots,e_n$ to $b_1,\dots, b_n$?

Consider $v = x$ on the basis $E=(e_1,\dots,e_n)$. We want to find $y$ such that $v = By$, i.e. $y = B^{-1} x$.

Then $Ax$ becomes $B(B^{-1}ABy)$, that is, $A \mapsto B^{-1} A B$.

\paragraph{Similar matrices} are matrices $A$ and $B$ s.t. $A = P^{-1} BP$.

\paragraph{Eigenbasis} Some basis changes turn $A$ into a nice form, e.g. $A \mapsto \operatorname{diag}(\lambda_1,\dots,\lambda_n)$.

A basis, in which $A$ is diagonal is called an eigenbasis of $A$.

\paragraph{Eigenvector} Any $v \neq 0$ s.t. $Av = \lambda v$ for some $\lambda$, called \textbf{eigenvalue}.


Correspondingly, eigenbasis is any basis formed by eigenvectors.

\textbf{Criterion}: $\lambda$ is an eigenvalue $\iff (A-\lambda I)v=0 \iff \det(A-\lambda I) =0$.

\paragraph{Characteristic polynomial} of the matrix $A$ is $p(\lambda)=\det(A-\lambda I)$.

Characteristic polynomial doesn't change when the basis is changed!

Each coefficient stays invariant under the change of the basis, in particular:

\begin{enumerate}
    \item $[\lambda^0] p(\lambda) = \lambda_1 \dots \lambda_n = \det A$, the determinant of $A$;
    \item $[\lambda^{n-1}]p(\lambda) = -(\lambda_1+\dots+\lambda_n) = -(A_{11}+\dots+A_{nn}) = -\operatorname{tr} A$, the trace of $A$.
\end{enumerate}

\paragraph{Complex numbers} Are numbers of form $a+bi \in \mathbb C$, where $i^2=-1$.

Algebraically closed field: Every polynomial can be factored in linear factors.

\paragraph{In-class exercises}

\begin{enumerate}
    \item For what values of $a,b,c \in \mathbb R$ is the determinant zero?
    $$
    A = \begin{bmatrix}
        0 & 1 & 0 & 4 & c \\
        a & 5 & 0 & 4 & -1 \\
        2 & 1 & b & -1 & -3 \\
        0 & -2 & 0 & 1 & 0 \\
        0 & -4 & 0 & 3 & 1
    \end{bmatrix}
    $$
    \item Find the determinant by performing Gaussian elimination:
    $$
    B = \begin{bmatrix}
        1 & 2 & -3 \\
        2 & 6 & 0 \\
        -1 & -2 & 2
    \end{bmatrix}
    $$
    \item How to find adjugate matrix?

    \item When $n$ is odd, any $n \times n$ matrix has a real eigenvalue, why?

    \item Example of a matrix that doesn't have eigenbasis.
\end{enumerate}

\end{document}
