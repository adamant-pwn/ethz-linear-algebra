\documentclass{article}
\usepackage[utf8]{inputenc}
\usepackage[T2A]{fontenc}
\usepackage[mytitle={Oleksandr Kulkov. ETH Zürich. Linear Algebra. Week 2},
            mylang=eng]{my_style}


\begin{document}

\paragraph{Topics of the week} Compute with matrices:  

\begin{enumerate}
    \item matrix-vector multiplication, column space, row space, rank;
    \item perform matrix multiplication, including matrix- vector, vector-matrix, scalar and outer product, distributivity, associativity.
\end{enumerate}

\paragraph{Basis} of a vector space is a maximal set of linearly independent vectors.

\textbf{Equivalent definition}: $\mathbf e_1, \dots, \mathbf e_n$ s.t. any $\mathbf v \in V$ can be \textit{uniquely} represented as
$$
\mathbf v = v_1 \mathbf e_1 + \dots + v_n \mathbf e_n.
$$

The numbers $(v_1,\dots,v_n)$ are called \textbf{coordinates} of $\mathbf v$ in $\mathbf e_1,\dots,\mathbf e_n$.

\textbf{Equivalent definition}: A minimal $\mathbf e_1, \dots, \mathbf e_n$ s.t. any $\mathbf v \in V$ has unique coordinates.

\paragraph{Linear transformation} between vector spaces $V$ and $W$ is a function $f : V \to W$, such that
\begin{align*}
\mathbf u + \mathbf v &\mapsto f(\mathbf u) + f(\mathbf v), \\
k \mathbf v  &\mapsto k f(\mathbf v).
\end{align*}

Let $\mathbf e_1,\dots, \mathbf e_n$ be the basis of $V$ and $\mathbf g_1, \dots, \mathbf g_m$ be the basis of $W$. Then,
$$
f(\mathbf v) = v_1 f(\mathbf e_1) + \dots + v_n f(\mathbf e_n).
$$

So, $f$ is fully defined by $f(\mathbf e_1), \dots, f(\mathbf e_n)$:

$$
f(\mathbf e_k) = f_{1k} \mathbf g_1 + \dots + f_{mk} \mathbf g_m.
$$

From this, we fully describe the linear map as
\begin{align*}
f(a_1 \mathbf e_1 + \dots + a_n \mathbf e_n) =& (a_1 f_{11} + \dots + a_n f_{1n}) \mathbf g_1 + \\
& (a_1 f_{21} + \dots + a_n f_{2n}) \mathbf g_2 + \\
& (a_1 f_{31} + \dots + a_n f_{3n}) \mathbf g_3 + \\
& \dots + \\
& (a_1 f_{m1} + \dots + a_n f_{mn}) \mathbf g_m \\
\end{align*}

This rewrites as
$$
\begin{cases}
b_1 = a_1 f_{11} + a_2 f_{12} + \dots + a_n f_{1n}, \\
b_2 = a_1 f_{21} + a_2 f_{22} + \dots + a_n f_{2n}, \\
\dots, \\
b_m = a_1 f_{m1} + a_2 f_{m2} + \dots + a_n f_{mn}.
\end{cases}
$$

Conventionally, this is written in matrix form as $b = Fa$:
$$
\begin{pmatrix}
b_1 \\ b_2 \\ \vdots \\ b_m
\end{pmatrix} = 
\begin{pmatrix}
f_{11} & f_{12} & \dots & f_{1n} \\
f_{21} & f_{22} & \dots & f_{2n} \\
\vdots & \vdots & \ddots & \vdots \\
f_{m1} & f_{m2} & \dots & f_{mn}
\end{pmatrix} \begin{pmatrix}
a_1 \\ a_2 \\ \vdots \\ a_n
\end{pmatrix}.
$$

\paragraph{Matrix product} is defined in a way that $AB$ corresponds to the linear map $\mathbf x \mapsto A(B(\mathbf x))$:

The columns of $A$ are $A (\mathbf e_1), \dots, A (\mathbf e_n) \implies$ the columns of $AB$ are $A(B(\mathbf e_1)), \dots, A(B(\mathbf e_n))$.

\paragraph{Interpretations of matrix product}
\begin{enumerate}
    \item $AB$ corresponds to applying $A$ to each column of $B$;
    \item $AB$ corresponds to applying $B$ to each row of $A$;
    \item $(AB)_{ij}$ is the dot product of the $i$-th row of $A$ and the $j$-th column of $B$.
\end{enumerate}

\paragraph{Dimension} of a vector space is the size of its basis. All bases have the same size.

\paragraph{Row and column spaces} of a matrix are spans of their rows/columns.

\paragraph{Matrix rank} is the dimension of its row/column space.

\paragraph{Lemma 2.21} Let $A$ be an $m \times n$ matrix. The following statements are equivalent:
\begin{enumerate}
    \item $\operatorname{rank} A \leq k$;
    \item There are vectors $\mathbf{v}_1,\dots,\mathbf{v}_k \in \mathbb R^m$ and $\mathbf{w}_1,\dots,\mathbf{w}_k \in \mathbb R^n$ such that $A = \sum\limits_{i=1}^k \mathbf{v}_i \mathbf{w}_i^\top$.
    \item $A = BC$, where $B \in \mathbb R^{m \times k}$ and $C \in \mathbb R^{k \times n}$ (rank decomposition).
\end{enumerate}

\paragraph{In-class exercises}

\begin{enumerate}
    \item Let $m \in \mathbb N_{\geq 2}$ be arbitrary and consider the $m \times m$ matrix
    $$
    \begin{bmatrix}
        a_{11} & a_{12} & \dots & a_{1m} \\
        a_{21} & a_{22} & \dots & a_{2m} \\
        \vdots & \vdots & \ddots & \vdots \\
        a_{m1} & a_{m2} & \dots & a_{mm}
    \end{bmatrix}
    $$
    with $a_{ij} = i + j$ for all $i,j \in \{1,2,\dots,m\}$. Determine the rank of $A$.
    \item Show that $(AB)^\top = B^\top A^\top$.
    \item A \textbf{conjugate} of $A$ is a matrix $A^*$ such that $\mathbf x \cdot A\mathbf y = A^* \mathbf x \cdot \mathbf y$. Find $A^*$.
    \item A matrix is \textbf{orthogonal} if it preserves distances. When is a matrix orthogonal?
\end{enumerate}

Only the first two exercises were actually covered in the class.

\end{document}
