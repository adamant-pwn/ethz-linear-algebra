\documentclass{article}
\usepackage[utf8]{inputenc}
\usepackage[T2A]{fontenc}
\usepackage[mytitle={Oleksandr Kulkov. ETH Zürich. Linear Algebra. Week 1},
            mylang=eng]{my_style}

\begin{document}

\paragraph{Topics of the week}

\begin{enumerate}
    \item Scalar product, length, cosine formula;
    \item Cauchy-Schwarz inequality, triangle inequality, perpendicular vectors;
    \item Define linear independence of vectors in three different ways;
    \item Work with the span of vectors.
\end{enumerate}

\paragraph{Real vector space} is a set $V$ with $+ : V \times V \to V$ and $\cdot : \mathbb R \times V \to V$, such that

\begin{enumerate}
    \item $V$ is a commutative group over $+$;
    \item Compatibility of scalar multiplication: $\alpha (\beta \vec x) = (\alpha \beta) \vec x$;
    \item Scalar identity: $1 \vec x = \vec x$;
    \item Distributivities: $\alpha(\vec x + \vec y) = \alpha \vec x + \alpha \vec y$ and $(\alpha + \beta) \vec x = \alpha \vec x + \beta \vec x$.
\end{enumerate}

Consequence: $0 \vec x = \vec 0$.

\paragraph{Inner product} is a map $\cdot : V \times V \to \mathbb R$, such that
\begin{enumerate}
    \item Symmetry: $\vec x \cdot \vec y = \vec y \cdot \vec x$;
    \item Linearity: $(a\vec x + b\vec y) \cdot \vec z = a (\vec x \cdot \vec z) + b (\vec y \cdot \vec z)$;
    \item Positive-definite: $\vec x \cdot \vec x \geq 0$ and $\vec x \cdot \vec x = 0 \iff \vec x = 0$.
\end{enumerate}

The standard scalar (dot) product of $(a_1,\dots,a_n)$ and $(b_1,\dots,b_n)$ is $a_1 b_1 + \dots + a_n b_n$.

\paragraph{Norm} is a map $\|\cdot \| : V \to \mathbb R$, such that
\begin{enumerate}
    \item Positive-definite: $\|\vec x\| \geq 0$ and $\|\vec x\| = 0 \iff \vec x = \vec 0$;
    \item Absolute homogeneity: $\|\lambda \vec x\| = |\lambda| \cdot \|\vec x\|$;
    \item Triangle inequality: $\|\vec x+\vec y\| \leq \|\vec x\| + \|\vec y\|$.
\end{enumerate}

Distance between two vectors: $\|\vec x - \vec y\|$. In inner product spaces, $\|\vec x\| = \sqrt{\vec x \cdot \vec x}$.

\paragraph{Cauchy-Schwarz inequality} For any $\vec x, \vec y \in V$, we have $$ |\vec x \cdot \vec y| \leq \|\vec x\| \cdot \|\vec y\|$$

Proof: Analyze $f(t) = \|\vec x - t \vec y \|^2$.

What does it mean geometrically?

\paragraph{Angle between vectors} Due to Cauchy-Schwarz inequality, we can define $\theta$ from

$$
\vec x \cdot \vec y = \|x\| \|y\| \cos \theta
$$

\paragraph{Triangle inequality} For any $\vec x, \vec y \in V$, we have $$\|x + y\| \leq \|x\| + \|y\|$$

\paragraph{In-class exercises}

\begin{enumerate}
    \item Let $\vec 0 \in \mathbb R^m$ denote the vector whose entries are all zero. We say that a set $L$ is a line in $\mathbb R^m$
if and only if there exists $\vec w \in \mathbb R^m$ with $\vec w \neq 0$ such that $L = \{\lambda \vec w : \lambda \in \mathbb R\}$. Let now $L$ be a line in $\mathbb R^m$ and let $\vec u$ be an arbitrary nonzero element of $L$. Prove $L = \{\lambda \vec u : \lambda \in \mathbb R\}$.
\item For two lines $L_1$ and $L_2$ in $\mathbb R^m$, prove that we have $L_1 \cap L_2 = \{\vec 0\}$ or $L_1 \cap L_2 =
L_1 = L_2$.
\item There are three points $A,B,C \in \mathbb R^n$. You need to go from $A$ to $B$ without getting closer to $C$ than $r$. What is the length of the shortest path that satisfies this?
\item Show that the following operations do not change the span:
\begin{enumerate}
    \item Multiplying any vector with a non-zero scalar;
    \item Adding one vector to another;
\end{enumerate}
\item We have a set of vectors $\vec v_1,\dots,\vec v_n$. How to find $\lambda_1,\dots,\lambda_n$ such that
$$
\lambda_1 \vec v_1 + \dots + \lambda_n \vec v_n = \vec v
$$
for a given $\vec v$? Assume that $\vec v_i \cdot \vec v_j = 0$ when $i \neq j$. What if it does not hold?
\item Given a set $\vec v_1, \dots, \vec v_n$, find $\vec u_1, \dots, \vec u_n$ that is orthogonal and has the same span.
\end{enumerate}

Only first two exercises were actually covered.

\end{document}
