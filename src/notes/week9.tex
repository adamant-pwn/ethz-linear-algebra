\documentclass{article}
\usepackage[utf8]{inputenc}
\usepackage[T2A]{fontenc}
\usepackage[mytitle={Oleksandr Kulkov. ETH Zürich. Linear Algebra. Week 9},
            mylang=eng]{my_style}


\begin{document}

\paragraph{Topics of the week} 

\begin{enumerate}
    \item Orthogonal vectors, Orthonormal vectors, Orthonormal bases;
    \item Orthogonal Matrices. Orthogonal matrices preserve norm and inner-product;
    \item Projections with orthonormal bases;
    \item Build an orthonormal basis with Gram-Schmidt (and show correctness of Gram-Schmidt)
    \item QR decomposition. Projections and least squares with QR decomposition.
\end{enumerate}

\paragraph{Orthogonal vectors} means pairwise orthogonal.


Let $e_1,\dots,e_n$ be orthogonal and $v = x_1 e_1 + \dots + x_n e_n$, then $e_i \cdot v = x_i (e_i \cdot e_i) \iff x_i = \frac{v \cdot e_i}{e_i \cdot e_i}$.

\paragraph{Orthonormal vectors} means orthogonal + each vector has a unit length.

Let $e_1,\dots,e_n$ be orthonormal and $v = x_1 e_1 + \dots + x_n e_n$, then $x_i = e_i \cdot v$.

\paragraph{Orthonormal basis} is a basis consisting of orthonormal vectors.

\paragraph{Gram matrix} For $v_1,\dots,v_n$:

$$
\Gamma(v_1,\dots,v_n) = \begin{bmatrix}
v_1 \cdot v_1 & v_1 \cdot v_2 & \dots & v_1 \cdot v_n \\
v_2 \cdot v_1 & v_2 \cdot v_2 & \dots & v_2 \cdot v_n \\
\vdots & \vdots & \ddots & \vdots \\
v_n \cdot v_1 & v_n \cdot v_2 & \dots & v_n \cdot v_n
\end{bmatrix} = A^\top A
$$

The columns of $A$ are orthogonal $\iff$ $A^\top A$ is diagonal.

The columns of $A$ are orthonormal $\iff A^\top A = I$.

\paragraph{Interpretation of projection} $A^\top b$ are scalar products of $b$ with all basis vectors of $C(A)$. Then, $(A^\top A)^{-1}$ recovers a vector of $C(A)$ from its dot products with the basis of $C(A)$. $(A^\top A)^{-1}$ allows to consider coordinate system in dot products with basis vectors, rather than in their linear combinations.

\paragraph{Dual basis} is the basis of $C(A)$ formed of the columns of $A (A^\top A)^{-1}$.

\paragraph{Orthogonal matrix} is a matrix that preserves distances: $\|Ax-Ay\| = \|x-y\| \iff A^\top A = I$.

In other words, matrix is orthogonal $\iff$ it maps $e_1,\dots,e_n$ into orthonormal basis.

\paragraph{Projection with orthonormal basis} Assume $A^\top A = I$, then the projection of $y$ on $Ax$ is $A A^\top y$.

\paragraph{Gram-Schmidt process} construct an orthonormal basis in $C(A)$ by subtracting projections.

\paragraph{QR decomposition} $A=QR$, where $Q$ is the result of Gram-Schmidt process, and $R = Q^\top A$.

\textbf{Note}: $AM = Q \implies A = QM^{-1} \implies R = M^{-1}$.

\paragraph{Project with QR decomposition} $\operatorname{proj}_{C(A)} (b) = QQ^\top b$.

\paragraph{In-class exercises} Consider the invertible matrices

$$
A = \begin{bmatrix}
    0 & 0 & 1 \\
    1 & 0 & 1 \\
    1 & 1 & 1
\end{bmatrix}, B = \begin{bmatrix}
    1 & 2 & 3 & 0 \\
    0 & 4 & 5 & 6 \\
    0 & 0 & 7 & 8 \\
    0 & 0 & 0 & 9
\end{bmatrix}
$$

\begin{enumerate}
    \item Apply the Gram-Schmidt process to the columns of $A$.
    \item Write down a $QR$-decomposition of $A$.
    \item Apply the Gram-Schmidt process to the columns of $B$.
    \item Is it always true that the Gram-Schmidt process on the columns of an upper triangular $n \times n$ matrix with non-zero diagonal entries yields the canonical basis $e_1,\dots,e_n$? Provide a proof or counterexample.
\end{enumerate}

\end{document}
