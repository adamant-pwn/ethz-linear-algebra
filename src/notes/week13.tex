\documentclass{article}
\usepackage[utf8]{inputenc}
\usepackage[T2A]{fontenc}
\usepackage[mytitle={Oleksandr Kulkov. ETH Zürich. Linear Algebra. Week 13},
            mylang=eng]{my_style}


\begin{document}

\paragraph{Topics of the week} 

\begin{enumerate}
    \item Spectral theorem: eigenvalues and eigenvectors of symmetric matrices
    \item Rayleigh quotients and their connection to eigenvalues of symmetric matrices
    \item Positive definite matrices, positive semidefinite matrices, Gram matrices
    \item Cholesky decomposition
    \item Singular value decomposition (SVD), derivation of singular value decomposition, connection to eigenvalue decomposition of $A^TA$ and $AA^T$, compact form of singular value decomposition
    \item Singular values, left singular vectors, right singular vectors.
\end{enumerate}

\paragraph{Invariant subspace} of $A : V \to V$ is a subspace $U \subset V$ s.t. $u \in U \implies A(u) \in U$.

\textbf{Note}: Any eigensubspace is invariant.

\textbf{Note}: If $U$ is an invariant subspace of $A$, then $U^\perp$ is an invariant subspace of $A^\top$:
$$
y \in U^\perp \iff \forall x \in U: x \cdot y = 0 \implies \forall x \in U: Ax \cdot y = 0 \iff \forall x \in U : x \cdot A^\top y = 0 \iff A^\top y \in U^\perp
$$
In particular, when $A=A^\top$, then $U^\perp$ is an invariant subspace of $A$.

\paragraph{Reduction} of $A$ on its invariant subspace $U$ is a transformation $A' : U \to U$, defined as $A'u = Au$.

\textbf{Note}: If $V = U_1 + U_2$ and $U_1 \cap U_2$, then there is a basis, in which
$$
A = \begin{bmatrix}
    A_1 & 0 \\ 0 & A_2
\end{bmatrix},
$$
where $A_1$ and $A_2$ are reductions on $U_1$ and $U_2$ correspondingly.

\paragraph{Spectral theorem} If $A = A^\top$, then $A$ has a real orthonormal eigenbasis.

\textbf{Proof}: Any eigenvalue of $A$ is real. Assume $Av = \lambda v$, then
$$
\lambda \|v\|^2 = Av \cdot v = v \cdot Av = \bar \lambda \|v\|^2
$$
Now, consider eigenvector $v$ and $U = \{kv : k \in \mathbb R\}$, then $U^\perp$ is invariant subspace of $A$. It means that we can consider the reduction of $A$ on $U^\perp$ and inductively find a basis of eigenvectors of $A$ that will all be orthogonal to $v$.

\textbf{Note}: It means that $A = V \Lambda V^\top = \sum\limits_{i=1}^n \lambda_i v_i v_i^\top$, where $V$ has eigenvectors as the columns.

\paragraph{Bilinear form} is a function $A(x, y)$ that is linear in $x$ and $y$.

\textbf{Note}: Any bilinear form can be represented in the coordinate form $A(x, y) = x^\top Ay$.

\paragraph{Quadratic form} is $B(x) = A(x, x)$, where $A(x, y)$ is a bilinear form s.t. $A(x, y) = A(y, x)$.

\textbf{Note}: Any quadratic form can be represented as $A(x) = x^\top Ax$, where $A = A^\top$.

\paragraph{Change of basis} In matrices, $A \mapsto X^{-1} A Y$, in bilinear forms $A \mapsto X^\top A Y$.

\textbf{Note}: In particular $A \mapsto V^\top AV$ for quadratic forms. Thus, only for orthonormal change of bases, $A$ changes in the same way as a quadratic form, and as a linear transformation.

\paragraph{Rayleigh quotient} Let $A = A^\top$, then the Rayleigh quotient of $x \neq 0$ is
$$
R(x) = \frac{x^\top Ax}{x^\top x}
$$
For Rayleigh quotient, it holds that $R(v_{\min}) = \lambda_{\min}$ and $R(v_{\max}) = \lambda_{\max}$.

\textbf{Proof}: Consider an orthonormal basis in which $A = \Lambda$:
$$
R(x) = \frac{\sum \lambda_i x_i^2}{\sum x_i^2} \in [\lambda_{\min}, \lambda_{\max}]
$$
\paragraph{Positive (semi)definite matrices} A symmetric matrix $A$ is:
\begin{enumerate}
    \item Positive definite if $x^\top Ax > 0$ for all $x \neq 0$;
    \item Positive semidefinite if $x^\top Ax \geq 0$ for all $x$;
\end{enumerate}

Equivalently, all eigenvalues are positive/non-negative.

\paragraph{Gram matrix} of $v_1,\dots,v_n$ is $V^\top V$, where $V = (v_1,\dots,v_n)$.

\textbf{Note}: Always positive semidefinite, and definite if linearly independent.

\paragraph{Cholesky decomposition} $A$ is positive semidefinite $\iff$ $A = C^\top C$, where $C$ is upper triangular.
$$
A=(V\sqrt\Lambda)(V\sqrt\Lambda)^\top = (QR)^\top (QR) = R^\top R
$$
Take $C = R$ from the $QR$ decomposition.

\paragraph{Singular value decomposition} is $A = U \Sigma V^\top = \sum\limits_{i=1}^n \sigma_i u_i v_i^\top$, where $U$ and $V$ are orthogonal, and $\Sigma$ is diagonal. Columns of $U$ and $V$ are left (right) singular vectors, and elements of $\Sigma$ are singular values.

\begin{enumerate}
    \item $U$ are the eigenvectors of $A A^\top$ and $V$ are the eigenvectors of $A^\top A$;
    \item Singular values are the square roots of eigenvalues of $A^\top A$ or $A A^\top$ (they're the same).
\end{enumerate}



\paragraph{In-class exercises}

\begin{enumerate}
    \item Let $S^\top = -S$. Prove that $-S^2$ is symmetric and positive definite.
    \item Let $A$ be a matrix of rank $1$. Find its singular values.
    \item Let $B_{ij}=1$. Prove that $A = I+B$ is invertible.
\end{enumerate}

\end{document}
