\documentclass{article}
\usepackage[utf8]{inputenc}
\usepackage[T2A]{fontenc}
\usepackage[mytitle={Oleksandr Kulkov. ETH Zürich. Linear Algebra. Week 3   },
            mylang=eng]{my_style}


\begin{document}

\paragraph{Topics of the week} 

\begin{enumerate}
    \item explain the CR decomposition;
    \item  linear transformations, visualizing linear transformations in 2d, properties of linear transformations, matrix representation of linear transformations;
    \item systems of linear equations, systems of linear equations with unique solutions.
\end{enumerate}

\paragraph{Actions on rows/columns} $AB$ for $A \in \mathbb R^{n \times m}$ and $B \in \mathbb R^{m \times k}$ is interpreted as:

\begin{enumerate}
    \item Applying matrix $A$ to all columns of $B$;
    \item Applying matrix $B$ to all rows of $A$.
\end{enumerate}

Let $A_i$ denote the $i$-th row and $A^j$ the $j$-th column:

\begin{enumerate}
    \item $(AB)_i$ is the linear combination of rows of $B_1,\dots,B_m$ with coefficients from $A_i$.
    \item $(AB)^j$ is the linear combination of columns of $A^1,\dots,A^k$ with coefficients from $B^j$.
\end{enumerate}

\paragraph{Rank decomposition} Let $A$ be an $m \times n$ matrix. The following statements are equivalent:
\begin{enumerate}
    \item $\operatorname{rank} A \leq k$;
    \item There are vectors $\mathbf{v}_1,\dots,\mathbf{v}_k \in \mathbb R^m$ and $\mathbf{w}_1,\dots,\mathbf{w}_k \in \mathbb R^n$ such that $A = \sum\limits_{i=1}^k \mathbf{v}_i \mathbf{w}_i^\top$.
    \item $A = CR$, where $C \in \mathbb R^{m \times k}$ and $R \in \mathbb R^{k \times n}$ (rank decomposition).
\end{enumerate}

When $k = \operatorname{rank} A$: Columns of $C$ form a basis in the column space. Rows of $R$ form a basis in the row space. Multiplying $R$ by $C$ on the left defines the way rows of $A$ are represented as linear combinations of the rows of $R$, similar for $C$ multiplied by $R$ on the right.

\paragraph{Systems of linear equations} Consider a system

\begin{align*}
&\begin{cases}
a_{11} x_1 + \dots + a_{1n} x_n = b_1, \\
a_{21} x_1 + \dots + a_{2n} x_n = b_2, \\
\dots, \\
a_{n1} x_1 + \dots + a_{nn} x_n = b_n.
\end{cases} &\iff& Ax = b &\iff&
\begin{pmatrix}
    a_{11} & a_{12} & \dots & a_{1n} \\
    a_{21} & a_{22} & \dots & a_{2n} \\
    \vdots & \vdots & \ddots & \vdots \\
    a_{n1} & a_{n2} & \dots & a_{nn}
\end{pmatrix} \begin{pmatrix}
    x_1 \\ x_2 \\ \vdots \\ x_n
\end{pmatrix} = \begin{pmatrix}
    b_1 \\ b_2 \\ \vdots \\ b_n
\end{pmatrix}& \\
&\begin{cases}
x_1 = b_1', \\
x_2 = b_2', \\
\dots, \\
x_n = b_n'.
\end{cases} &\iff& A^{-1} A x = A^{-1} b &\iff& \begin{pmatrix}
    1 & 0 & \dots & 0 \\
    0 & 1 & \dots & 0 \\
    \vdots & \vdots & \ddots & \vdots \\
    0 & 0 & \dots & 1
\end{pmatrix} \begin{pmatrix}
    x_1 \\ x_2 \\ \vdots \\ x_n
\end{pmatrix} = \begin{pmatrix}
    b_1' \\ b_2' \\ \vdots \\ b_n'
\end{pmatrix}&
\end{align*}

\paragraph{In-class exercises}

\begin{enumerate}
    \item Consider $T : \mathbb R^n \to \mathbb R$ for $n > 0$ defined as $T(x) = \sum\limits_{k=1}^n k x_k$.
    Prove that $T$ is a linear transformation.
    \item Consider $T : \mathbb R^n \to \mathbb R$ for $n \geq 2$ defined as $T(x) = \sum\limits_{k=1}^n (x_k)^k$.
    Is $T(x)$ a linear transformation?
\end{enumerate}

\end{document}
