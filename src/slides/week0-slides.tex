\documentclass[10pt]{beamer}

\usetheme[progressbar=head]{moloch}

\usepackage[utf8]{inputenc}

% Essential packages
\usepackage{listings}
\usepackage{mathtools}
\usepackage{tikz}
\usepackage{svg}
\usepackage{booktabs}
\usepackage{ccicons}
\usepackage{pgfplots}
\usepgfplotslibrary{dateplot}
\usepackage{xspace}
\usepackage[dvipsnames]{xcolor}
\usepackage{hyperref}

\hypersetup{
    colorlinks=true,
    linkcolor=blue,
    filecolor=magenta,      
    urlcolor=cyan,
    pdftitle={Overleaf Example},
    pdfpagemode=FullScreen,
    }

% Configure Moloch theme settings
\molochset{
    sectionpage=none,
    block=fill
}

\usefonttheme{professionalfonts}
\usepackage{bm}

% Progress bar

\setbeamertemplate{page number in head/foot}[totalframenumber]

\makeatletter
\setlength{\moloch@progressinheadfoot@linewidth}{1em}

\addtobeamertemplate{headline}{% 
    \begin{beamercolorbox}[colsep=1.5pt]{upper separation line head}
    \end{beamercolorbox}
    \begin{beamercolorbox}{section in head/foot}
        \vskip2pt\insertnavigation{\paperwidth}\vskip2pt
    \end{beamercolorbox}%
    \begin{beamercolorbox}[colsep=1.5pt]{lower separation line head}
    \end{beamercolorbox}
    \par
}{}

\setbeamertemplate{footline}{
    \begin{tikzpicture}[remember picture,overlay]
      \node[anchor=north east] at ([yshift=-5.5ex]current page.north east) {\usebeamercolor[fg]{page number in head/foot}\usebeamertemplate{page number in head/foot}};
    \end{tikzpicture}
}

\def\beamer@writeslidentry{\clearpage\beamer@notesactions}  
\makeatother

\setbeamercolor{section in head/foot}{fg=black, bg=white}

\setbeamercolor{page number in head/foot}{fg=mLightBrown}



% Encoding and localization
\usepackage[utf8]{inputenc}
\usepackage[T2A]{fontenc}
\usepackage[english]{babel}

% Code listing settings
\definecolor{mauve}{rgb}{0.58,0,0.82}
\definecolor{dkgreen}{rgb}{0,0.4,0}
\definecolor{gray}{rgb}{0.5,0.5,0.5}
\definecolor{lightgray}{rgb}{0.95,0.95,0.95}

\lstset{
  language=Python,
  basicstyle=\fontsize{5}{6}\ttfamily,
  numbers=left,
  stepnumber=1,
  numbersep=0.7em,
  backgroundcolor=\color{lightgray},
  showspaces=false,
  showstringspaces=false,
  showtabs=false,
  frame=single,
  rulecolor=\color{black},
  tabsize=2,
  breaklines=true,
  breakatwhitespace=false,
  identifierstyle=\color{blue!25!black},  
  keywordstyle=\color{blue!90!black},
  commentstyle=\color{dkgreen},
  stringstyle=\color{mauve},
  escapeinside={\`}{\`}, 
  escapebegin=\color{gray!50!black}\footnotesize,
  keepspaces=true,
  columns=fullflexible
}

% Presentation title and metadata
\title{W0: Vectors and linear combinations}
\date{}
\author{Oleksandr Kulkov}
\institute{ETH Zürich}
\titlegraphic{\hfill\includesvg[height=0.5cm]{../../img/svg/eth}}


\begin{document}

\maketitle

\section{Intro}
\begin{frame}{Group info}
    \textbf{Group}: 24

    \textbf{Location}: CHN G 46
    
    \textbf{Teaching assistant}: Oleksandr Kulkov

    \textbf{Language}: English

    \textbf{Focus group}\footnote{More thorough and detailed explanation of basics}: Yes

    \textbf{Materials}: \small\url{https://github.com/adamant-pwn/ethz-linear-algebra}

    Please don't hesitate to interrupt at any moment if have any questions!
\end{frame}

\begin{frame}{Course info}
\begin{itemize}
    \item Exam structure:
    \begin{itemize}
        \item \textbf{Calculations}: Solve standard compute tasks (answer-only);
        \item \textbf{Proofs}: Need to justify anything that wasn't in \textit{lectures};
        \item \textbf{Multi-choice questions}: Pick the only correct option;
    \end{itemize}
    \item Weekly bonus tasks: up to 0.25 extra points to grade;
    \item Hand-ins: One exercise per week, get feedback;
\end{itemize}
\end{frame}

\section{Scalars}
\begin{frame}{Scalars}
    Recall ``standard'' sets of numbers:

    \begin{itemize}
        \item $\mathbb N$: \textbf{Natural} numbers;
        \item $\mathbb Z$: \textbf{Integer} numbers;
        \item $\mathbb Q$: \textbf{Rational} numbers;
        \item $\mathbb R$: \textbf{Real} numbers;
    \end{itemize}

    Represent \textit{quantity}, typically called the \textbf{scalars}\footnote{because of their role in the ``scaling''}.
\end{frame}

\section{Tuples}
\begin{frame}{Tuples}
    $\mathbb R^m$ is the set of \textbf{tuples}: $\mathbb R^m = \{(x_1,\dots,x_m) : x_i \in \mathbb R\}$:

    \begin{itemize}
        \item $\mathbb R^2 = \{(x, y) : x, y \in \mathbb R\}$ -- points of a 2D \textbf{plane};
        \item $\mathbb R^3 = \{(x, y, z): x, y, z \in \mathbb R\}$ -- points of a 3D \textbf{space};
    \end{itemize}

    The \textbf{power} symbol corresponds to the \textbf{Cartesian product}:
    $$
    A \times B = \{(a, b) : a \in A \text{ and } b \in B\}
    $$
    We may use \textit{indices} to refer to the \textit{components} of a tuple.
\end{frame}

\section{Vectors}
\begin{frame}{Vectors}
    \textbf{Real vectors}\footnote{\textit{usually} denoted by lowercase \textbf{bold} letters} are elements of a set $V$ with:

    \begin{itemize}
        \item \textbf{Addition}: $\mathbf{u} + \mathbf{v}$ for $\mathbf{u}, \mathbf{v} \in V$;
        \item \textbf{Scaling}: $k \mathbf{v}$ for $k \in \mathbb R$ and $\mathbf{v} \in V$;
    \end{itemize}

    Addition and scaling are collectively called \textbf{linear operations}\footnote{hence, they are the subject of \textit{linear} algebra}.

    In most general case, addition and scaling are \textbf{axiomatic}\footnote{defined by their properties}.

    In our course, we will mostly work with $V = \mathbb R^m$.
\end{frame}

\begin{frame}{Column vectors}
    For $V = \mathbb R^m$, vectors are \textit{usually} represented as columns:
    $$
    \mathbf v = \begin{bmatrix}
        v_1 \\ v_2 \\ \vdots \\ v_m
    \end{bmatrix},
    $$
    and the addition and scaling are \textbf{component-wise}:
    \begin{itemize}
        \item $(\mathbf u + \mathbf v)_i = u_i + v_i$;
        \item $(k \mathbf v)_i = k v_i$;
    \end{itemize}

    This is geometry-motivated for $m=2$ and $m=3$.
\end{frame}

\section{Geometry}
\begin{frame}{Vector geometry}
    In Physics, vectors are objects with \textit{direction} and \textit{magnitude}.
\end{frame}

\end{document}